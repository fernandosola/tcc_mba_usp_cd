



No ambiente de negócios, o gerenciamento de reclamações de clientes tem se tornado um fator crítico de sucesso para manter um bom relacionamento e a fidelidade de clientes, conforme afirmam \citeonline{Coussement2008}. No ambiente da administração pública, onde o cidadão é o cliente, a prestação de serviços à população deve ser guiada por princípios como qualidade, eficiência e economicidade. Assim, diversos canais são disponibilizados aos cidadãos para entender melhor suas necessidades ou mesmo receber informações importantes sobre anomalias ou ilicitudes identificadas.

Centenas de denúncias são enviadas diariamente através da Plataforma Integrada de Ouvidoria e Acesso à Informação (Fala.BR). Para que possam ser apuradas, primeiramente elas precisam ser triadas por uma equipe da \sigla{OGU}{Ouvidoria-Geral da União}, área integrante da \sigla{CGU}{Controladoria-Geral da União}, responsável por exercer as competências de órgão central do Sistema de Ouvidoria do Poder Executivo federal. O processo de triagem consiste em avaliar se há o mínimo de informações necessárias presente na denúncia e, caso haja, encaminhar a denúncia para que uma área competente realize a apuração dos fatos denunciados.

Em média, apenas 30\% das denúncias que chegam são consideradas aptas pelos especialistas da Ouvidoria. Com isso, uma quantidade enorme de esforço é gasta pela equipe em denúncias que não possuem o mínimo de informação para qualquer tipo de apuração. O grande volume de denúncias é um obstáculo que exige uma grande equipe trabalhando no processo de triagem. 

Assim, a criação de um modelo de classificação que atribua o grau de risco de aptidão para uma denúncia permite direcionar melhor os esforços das equipes responsáveis pelo processo de triagem. O objetivo desta pesquisa é propor uma metodologia que permita a implementação deste modelo nos processo de triagem de  denúncias da OGU. 

O dados de denúncias recebidos através do sistema Fala.BR possuem poucos campos estruturados. A essência da denúncia está contida no texto escrito pelo cidadão e nos diversos anexos que, por vezes, são enviados em conjunto. Para uma pessoa, a análise de uma denúncia, na maioria das vezes é uma tarefa simples. Basicamente, o analista procura por elementos no texto que possam ser confirmados por consultas em sistemas oficiais como nomes de pessoas, números de cps, números de cnpj, endereços, órgãos, contratos, convênios, valores, além de outros elementos que possam remeter a uma certa gravidade sobre o problema exposto pelo cidadão. Entretanto, fazer este tipo de análise computacionalmente não é uma tarefa trivial. O \sigla{PLN}{Processamento de linguagem natural} é uma subárea da Ciência da Computação cujo o foco é a extração de informação. Este campo é vasto e é possível utilizar abordagens estatísticas como em \citeonline{Manning1999} ou combiná-las com abordagens simbólicas 


tial importance either because they suggest ways of combining statistical approaches with symbolic approaches (as in the regular-expression
post-filtering of collocations in (Justeson and Katz 1995b)) or because
the insights they offer can often be expressed in a statistical framework


Como abordarei o problema?

Para isso, será necessário compreender as diferentes formas de extrair informações de textos


Tema da Pesquisa ???
Delimitação do assunto ???