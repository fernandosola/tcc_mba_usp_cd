% Comando simples para exibir comandos Latex no texto
\newcommand{\comando}[1]{\textbf{$\backslash$#1}}

No ambiente de negócios, o gerenciamento de reclamações de clientes tem se tornado um fator crítico de sucesso para manter um bom relacionamento e a fidelidade de clientes, conforme afirmam Coussement e Van den Poel (2008). No ambiente da administração pública, onde o cidadão é o cliente, a prestação de serviços à população deve ser guiada por princípios como qualidade, eficiência e economicidade. Assim, diversos canais são disponibilizados aos cidadãos para entender melhor suas necessidades ou mesmo receber informações importantes sobre anomalias ou ilicitudes identificadas.
Centenas de denúncias são enviadas diariamente através da Plataforma Integrada de Ouvidoria e Acesso à Informação (Fala.BR). Para que possam ser apuradas, primeiramente elas precisam ser triadas por uma equipe da \sigla{OGU}{Ouvidoria-Geral da União}, área integrante da Controladoria-Geral da União (CGU), responsável por exercer as competências de órgão central do Sistema de Ouvidoria do Poder Executivo federal. O processo de triagem consiste em avaliar se há o mínimo de informações necessárias presente na denúncia e, caso haja, encaminhar a denúncia para que uma área competente realize a apuração dos fatos denunciados.
Em média, apenas 30\% das denúncias que chegam são consideradas aptas pelos especialistas da Ouvidoria. Assim, uma quantidade enorme de esforço é gasta pela equipe em denúncias que não possuem o mínimo de informação para qualquer tipo de apuração.  O grande volume de denúncias é um obstáculo que exige uma grande equipe trabalhando no processo de triagem. Assim, a criação de um modelo de classificação que atribua o grau de risco de aptidão para uma denúncia permite direcionar melhor os esforços das equipes responsáveis pelo processo de triagem.

