\sigla*{TIC}{Tecnologias da Informação e Comunicação}

A utilização de reclamações e denúncias como subsídio para melhorar a qualidade dos serviços oferecidos pelo Estado é uma prática comum. Com a crescente evolução de técnicas e capacidade computacional nos últimos anos, é natural observar um aumento do seu emprego na automatização de processos relacionados à classificação e análise desse conjunto imenso de informações não estruturadas que chegam aos órgãos públicos constantemente. Esse entendimento de que associar tecnologia, serviços governamentais e participação social é uma tendência, é afirmado em \citeonline{} é enfatizado em \citeonline[p.1]{Chun2010}:

\begin{citacao}
A revolução nas tecnologias da informação e comunicação (TIC) vem mudando não apenas o cotidiano das pessoas, mas também as interações entre governos e cidadãos. O governo digital ou o governo eletrônico começou como uma nova forma de organização pública que suporta e redefine as informações novas e existentes, as comunicações e as interações relacionadas às transações com as partes interessadas (por exemplo, cidadãos e empresas) por meio das TIC, especialmente por meio das tecnologias da Internet e da Web, com o objetivo de melhorar o desempenho e os processos do governo.
\end{citacao}

Na Alemanha explorou-se a ideia de que seria possível complementar modelos frequentemente utilizados em investigação forense eleitoral com informações de denúncias de cidadãos a respeito das eleições. Tais modelos utilizam métodos estatísticos, geralmente baseados em contagem de votos e na quantidade de eleitores elegíveis, para avaliar se os resultados de eleições refletem as intenções dos eleitores ou, ainda, avaliar se há indícios de fraudes no processo eleitoral. O uso dessas denúncias é uma forma de incorporar informação contextual e isso pode ser útil para avaliar e refinar os modelos existentes \cite{mebane2016frauds}.

Em \citeonline{liyanage2018matters}, propõe-se uma forma automatizada de priorização de problemas urbanos reportados por cidadãos. O processo sugerido utiliza informações textuais, imagens e votos dados pelos cidadãos em cada problema reportado. Com estas informações o modelo proposto atribui uma probabilidade de o problema ser de alta importância permitindo assim, uma ordenação por prioridade.




% citar o trabalho da patricia 



% evidenciar por que os trabalhos citados nao atendem no que se propoe e 