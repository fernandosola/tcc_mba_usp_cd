
A utilização de reclamações e denúncias como subsídio para melhorar a qualidade dos serviços oferecidos pelo Estado é uma prática comum. Com a crescente evolução de técnicas e capacidade computacional nos últimos anos, é natural observar um aumento do seu emprego na automatização de processos relacionados à classificação e análise desse conjunto imenso de informações não estruturadas que chegam aos órgãos públicos constantemente. 

Na Alemanha explorou-se a ideia de que seria possível complementar modelos frequentemente utilizados em investigação forense eleitoral com informações de denúncias de cidadãos a respeito das eleições. Tais modelos utilizam métodos estatísticos, geralmente baseados em contagem de votos e na quantidade de eleitores elegíveis, para avaliar se os resultados de eleições refletem as intenções dos eleitores ou, ainda, avaliar se há indícios de fraudes no processo eleitoral. Conforme afirmam os autores do estudo, o uso dessas denúncias é uma forma de incorporar informação contextual e isso pode ser útil para avaliar e refinar os modelos existentes. Empregou-se diversas técnicas para tratar os textos das denúncias e por fim criou-se uma matriz de frequências de termos por documento. Com isso, foi possível criar um classificador que pudesse separar as denúncias de acordo com problema ou incidente endereçado por ela. A informação de classificação foi, então, utilizada para compor o modelo final proposto\cite{mebane2016frauds}.

