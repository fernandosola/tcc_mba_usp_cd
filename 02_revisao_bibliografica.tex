
A utilização de reclamações e denúncias como subsídio para melhorar a qualidade dos serviços oferecidos pelo Estado é uma prática comum. Com a crescente evolução de técnicas e capacidade computacional nos últimos anos, é natural observar um aumento do seu emprego na automatização de processos relacionados à classificação e análise desse conjunto imenso de informações não estruturadas que chegam aos órgãos públicos constantemente. 

Na Alemanha explorou-se a ideia de que seria possível complementar modelos frequentemente utilizados em investigação forense eleitoral com informações de denúncias de cidadãos a respeito das eleições. Tais modelos utilizam métodos estatísticos, geralmente baseados em contagem de votos e na quantidade de eleitores elegíveis, para avaliar se os resultados de eleições refletem as intenções dos eleitores ou, ainda, avaliar se há indícios de fraudes no processo eleitoral. O uso dessas denúncias é uma forma de incorporar informação contextual e isso pode ser útil para avaliar e refinar os modelos existentes \cite{mebane2016frauds}.


\begin{citacao}
The revolution in information and communication technologies (ICT) has been changing not only the daily lives of people but also the interactions between governments and citizens. The digital government or electronic government (e-government) has started as a new form of public organization that supports and redefines the existing and new information, communication and transaction-related interactions with stakeholders (e.g., citizens and businesses) through ICT, especially through the Internet and Web technologies, with the purpose of improving government performance and processes \cite{}.
\end{citacao}


